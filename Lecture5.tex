\PassOptionsToPackage{table}{xcolor}
\documentclass[notitlepage,12pt]{article}

% \title{\vspace{-3.0cm}Notes}
% \date{}
% \author{Naren}	

\usepackage{marginnote}
\usepackage[center]{titlesec}
\usepackage[paper=a4paper,top=1in, left=1in, right=1in, bottom=1in,  heightrounded,
marginparwidth=1in, marginparsep=3mm]{geometry}
\usepackage[dvipsnames]{xcolor}
\usepackage{amsthm, amssymb, amsfonts, amsmath, mathtools}
\usepackage{graphicx, tikz, hyperref, enumitem, mathtools, mathrsfs, tikz-cd, adjustbox }
\usetikzlibrary{calc,shapes}
\usetikzlibrary{matrix}
\usepackage{multicol}
\usepackage{url}
\usepackage{nameref}
\usepackage{wrapfig}
\usepackage{faktor}
\usepackage{bbold}
\usepackage{float}
\usepackage{todonotes}
\usepackage{parskip}
\usepackage{cleveref}
\usepackage{tikz}

\newcommand{\C}{\mathbb{C}}
\newcommand{\R}{\mathbb{R}}
\newcommand{\Q}{\mathbb{Q}}
\newcommand{\Z}{\mathbb{Z}}
\newcommand{\V}{\mathcal{V}}

\newcommand{\N}{\mathbb{N}}
\newcommand{\E}[2]{E^{#1}_{#2}}
\newcommand{\cf}{ \Gamma\mathrm{Hom}(\pi^*E_{\infty},\pi^*E_0)}
\newcommand{\Eo}{E_0}
\newcommand{\Ef}{E_{\infty}}
\newcommand{\lin}{\operatorname*{lin}}

\hypersetup{colorlinks=true, linkcolor=Red, citecolor=RedOrange, urlcolor=ForestGreen}

\usetikzlibrary{matrix}

\theoremstyle{definition}
\newtheorem{theorem}{Theorem}[section]
\newtheorem{corollary}[theorem]{Corollary}
\newtheorem{lemma}[theorem]{Lemma}
\newtheorem{definition}[theorem]{Definition}
\newtheorem{example}[theorem]{Example}
\newtheorem{remark}[theorem]{Remark}
\newtheorem*{claim}{Claim}
\newtheorem{proposition}[theorem]{Proposition}

\usepackage[skins]{tcolorbox}

% \renewcommand{\thetheorem}{\arabic{theorem}}
% \renewcommand{\theexample}{\arabic{example}}
\makeatletter
\newtheoremstyle{para}
  {\topsep}   % ABOVESPACE
  {\topsep}   % BELOWSPACE
  {\upshape}  % BODYFONT
  {0pt}       % INDENT (empty value is the same as 0pt)
  {\bfseries} % HEADFONT
  {.}         % HEADPUNCT
  {5pt plus 1pt minus 1pt} % HEADSPACE
  {\thmnumber{#2}\@ifnotempty{#3}{ \thmnote{#3}}} % CUSTOM-HEAD-SPEC
\makeatother
\theoremstyle{para}{\normalfont}
\newtheorem{para}[theorem]{\normalfont}
\newtheorem*{para*}{para}
\newcommand\descitem[1]{\item{\bfseries #1}\\}
\DeclareMathOperator{\sq}{Sq}
\usepackage{abstract}
\renewcommand{\abstractname}{}    % clear the title
\renewcommand{\absnamepos}{empty}
\usepackage{spectralsequences}
\usepackage[T1]{fontenc}
% \usepackage[tracking]{microtype}

\usepackage[sc,osf]{mathpazo}   % With old-style figures and real smallcaps.
\linespread{1.025}              % Palatino leads a little more leading

%  % Euler for math and numbers
%  \usepackage[euler-digits,small]{eulervm}
% \AtBeginDocument{\renewcommand{\hbar}{\hslash}}
%\usepackage{kpfonts}
\makeatletter
\newcommand*{\toccontents}{\@starttoc{toc}}
\makeatother

\usepackage{soul}
\newcommand{\mathcolorbox}[2]{\colorbox{#1}{$\displaystyle #2$}}


\usepackage[T1]{fontenc}
\DeclareMathOperator{\colim}{colim}

\begin{document}
\begin{titlepage}
    \title{Sequential Spectra\\
    Lecture 5
}
\author{  }
	\maketitle
	\thispagestyle{empty}
    	\toccontents
\end{titlepage}

We work in the category $\mathcal{T}$ of based( compactly generated weak Hausdorff ) spaces  and basepoint preserving maps. 
\begin{definition}
    A (sequential) spectrum $X$ is a sequence of spaces $X_n$ for $n\geq 0$ and structure maps $\sigma_n:\Sigma X_n=X_n\wedge S^1\to X_{n+1 }$. 
  
    A map of spectra $f:X\to Y$ is a sequence of maps $f_n:X_n\to Y_n$ such that eacg square of the following form commutes \marginnote{$S^1$ commutes with $\wedge$}
    \[
        \begin{tikzcd}
\Sigma X_n \arrow[r, "\sigma_n"] \arrow[d, "\Sigma f_n"'] & X_{n+1} \arrow[d, "\Sigma f_{n+1}"] \\
\Sigma Y_n \arrow[r, "\sigma_n"']                         & Y_{n+1}                            
\end{tikzcd}
    \]

    The category of spectra and the maps described above is denoted by $\mathcal{S}p$.
\end{definition}
Note that the bonding maps $\sigma_i:\Sigma X_i\to \Sigma_{i+1 }$ have adjoints $\tilde{\sigma}_i:\Omega X_i\to \Omega X_{i+1}$ and we could equivalently define a spectrum using the adjoint bonding maps. 

\begin{definition}
    The degree $k\in \mathbb{Z}$ stable homotopy group of a spectrum $X$ is the abelian group 
    \[
        \pi_k(X)=\colim_n \pi_{k+n}(X_n)
    \]Here $\pi_{k+n }(X_n)\to \pi_{k+n+1 (X_{n+1})}$ maps the homotopy class of $\phi:S^{k+n } \to X_n$ to the class of $\sigma(S^1\wedge \phi):S^1\wedge S^{k+n } \to X_{n+1 }$.
\end{definition}

\begin{definition}
    A map of spectra $f:X\to Y$ is a stable equivalence ($\pi_*$ isomorphism ) if the induced map $f_*:\pi_*(X)\to \pi_*(Y)$ is an isomorphism. 
\end{definition}

Consider $\mathcal{W}\subset \mathcal{S}p$ to be the wide subcategory of stable equivalences. The stable homotopy category $\mathrm{ho } \mathcal{S}p$ is the localization of $\mathcal{S}p$ at $\mathcal{W}$. The functor 
\[
    \mathcal{S}p\to \mathcal{S}p[\mathcal{W}^{-1 } ]=\mathrm{ho } \mathcal{S}p
\] sends stable equivalences to isomorphisms. \marginnote{Initial functor?}

\begin{example}
    \begin{enumerate}
    \item The suspension spectrum $\Sigma^{\infty } X$ of a space $X$ is the spectrum, with $(\Sigma^{\infty } X)_n=\Sigma^nX$, with the bonding maps $\operatorname*{id }:S^1\wedge \Sigma^iX\to \Sigma^{i+1 } X$. The suspension spectrum for an unpointed space $X$ is obtained by adding a disjoint basepoint: $\Sigma^{\infty } _+X\coloneqq \Sigma^{\infty } (X_+)$. \\
    The homotopy groups of $\Sigma^{\infty } X$ are the stable homotopy groups of the space $X$. 
        \item An important example of a suspension spectrum is the \textit{sphere spectrum} $$\mathbb{S}\coloneqq \Sigma^{\infty } S^0$$ This is the sequence of spheres $S^0,S^1,S^2,\dots$ \\
        The homotopy groups of the sphere spectrum are called the \textit{stable stems }. They are the stable homotopy groups of $S^0$, often written $\pi_i^s\coloneqq \pi_i \mathbb{S}$.
    \item \label{2}The zero spectrum $\Sigma^{\infty}*$ is the suspension spectrum of a point, with the bonding maps 
    \[
        \Sigma(*)\cong * \to *
    \] 
    Every spectrum $X$ admits a unique maps of spectra $*\to X\to *$. $X$ is weakly contractible if one (equivalently both) of these maps are stable equivalences. \\
    The suspension spectrum has all of its homotopy groups zero. 

    Given two spectra $X$ and $Y$, the zero map $X\to Y$ is the unique map of spectra that factors through $*$
    \[
        \begin{tikzcd}
X_n \arrow[rd] \arrow[rr] &              & Y_n \\
                          & * \arrow[ru] &    
\end{tikzcd}
    \]
    
    \item\label{3} Eilenberg-Maclane spaces are infinite loopspaces. We can define teh Eilenberg-Maclane spectrumm $HA$ by 
    \[
        (HA)_n=K(A,n)
    \] with bonding maps adjoing to the canonical equivalences $K(A,n)\simeq \Omega K(A,n+1)$. We use the letter $H$ because the infinite loopspace represents cohomology with coefficients in $A$. \marginnote{todo}

    The homotopy grops of $HA$ are given by 
    \[
        \pi_i HA=\begin{cases}
            A&i=0\\
            0&i\neq 0
        \end{cases}
    \]  
    \item \label{4} The complex $K$ theory spectrum $KU$
    \[
        KU_n=\begin{cases}
            \mathbb{Z}\times BU&n \text{ even}\\
            U &n \text{ odd}
        \end{cases}
    \]The structure maps adjoint to the equivalences 
    \[
        \Omega U\simeq \mathbb{Z}\times BU \quad \Omega(\mathbb{Z}\times BU)\simeq U
    \] 
    The homotopy groups of $KU$ are given by \marginnote{Bott element\\generator    }
    \[
        \pi_iKU=\begin{cases}
            \mathbb{Z}&i \text{ even}\\
            0& i \text{ odd}
        \end{cases}
    \]\end{enumerate}
\end{example}
\begin{definition}
    An $\Omega-$ spectrum (or fibrant spectrum) is a spectrum $X$ in which the adjunct bonding maps 
    \[
        X_n\xrightarrow[ ]{\simeq } \Omega X_{n+1 }
    \] are weak homotopy equivalences. 
\end{definition} The zeroeth space $X_0$ of an $\Omega$ -spectrum is an infinite loop space. 
Examples \ref{2},\ref{3},\ref{4} above are $\Omega$-spectra. 


\end{document}