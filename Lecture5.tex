\PassOptionsToPackage{table}{xcolor}
\documentclass[notitlepage,12pt]{article}
\title {}
\date{}

\usepackage{marginnote}
\usepackage[center]{titlesec}
\usepackage[paper=a4paper,top=1in, left=1in, right=1in, bottom=1in,  heightrounded,
marginparwidth=1in, marginparsep=3mm]{geometry}
\usepackage[dvipsnames]{xcolor}
\usepackage{amsthm, amssymb, amsfonts, amsmath, mathtools}
\usepackage{graphicx, tikz, hyperref, enumitem, mathtools, mathrsfs, tikz-cd, adjustbox }
\usetikzlibrary{calc,shapes}
\usetikzlibrary{matrix}
\usepackage{multicol}
\usepackage{url}
\usepackage{nameref}
\usepackage{wrapfig}
\usepackage{faktor}
\usepackage{bbold}
\usepackage{float}
\usepackage{todonotes}
\usepackage{parskip}
\usepackage{cleveref}
\usepackage{tikz}

\newcommand{\C}{\mathbb{C}}
\newcommand{\R}{\mathbb{R}}
\newcommand{\Q}{\mathbb{Q}}
\newcommand{\Z}{\mathbb{Z}}
\newcommand{\V}{\mathcal{V}}

\newcommand{\N}{\mathbb{N}}
\newcommand{\E}[2]{E^{#1}_{#2}}
\newcommand{\cf}{ \Gamma\mathrm{Hom}(\pi^*E_{\infty},\pi^*E_0)}
\newcommand{\Eo}{E_0}
\newcommand{\Ef}{E_{\infty}}
\newcommand{\lin}{\operatorname*{lin}}

\hypersetup{colorlinks=true, linkcolor=Red, citecolor=RedOrange, urlcolor=ForestGreen}

\usetikzlibrary{matrix}

\theoremstyle{definition}
\newtheorem{theorem}{Theorem}[section]
\newtheorem{corollary}[theorem]{Corollary}
\newtheorem{lemma}[theorem]{Lemma}
\newtheorem{definition}[theorem]{Definition}
\newtheorem{example}[theorem]{Example}
\newtheorem{remark}[theorem]{Remark}
\newtheorem*{claim}{Claim}
\newtheorem{proposition}[theorem]{Proposition}

\usepackage[skins]{tcolorbox}

\renewcommand{\thetheorem}{\arabic{theorem}} %remove section counter 
\renewcommand{\theexample}{\arabic{example}}
\makeatletter
\newtheoremstyle{para}
  {\topsep}   % ABOVESPACE
  {\topsep}   % BELOWSPACE
  {\upshape}  % BODYFONT
  {0pt}       % INDENT (empty value is the same as 0pt)
  {\bfseries} % HEADFONT
  {.}         % HEADPUNCT
  {5pt plus 1pt minus 1pt} % HEADSPACE
  {\thmnumber{#2}\@ifnotempty{#3}{ \thmnote{#3}}} % CUSTOM-HEAD-SPEC
\makeatother
\theoremstyle{para}{\normalfont}
\newtheorem{para}[theorem]{\normalfont}
\newtheorem*{para*}{para}
\newcommand\descitem[1]{\item{\bfseries #1}\\}
\DeclareMathOperator{\sq}{Sq}
\usepackage{abstract}
\renewcommand{\abstractname}{\color{brown}Lecture 5}    % clear the title
\renewcommand{\absnamepos}{empty}
\usepackage{spectralsequences}
\usepackage[T1]{fontenc}

% \usepackage[sc,osf]{mathpazo}   % With old-style figures and real smallcaps.
% \linespread{1.025}              % Palatino leads a little more leading
 \usepackage[bitstream-charter]{mathdesign}

% \usepackage{libertine}
%  % Euler for math and numbers
%  \usepackage[euler-digits,small]{eulervm}
% \AtBeginDocument{\renewcommand{\hbar}{\hslash}}

\makeatletter
\newcommand*{\toccontents}{\@starttoc{toc}}
\makeatother

\usepackage{soul}
\newcommand{\mathcolorbox}[2]{\colorbox{#1}{$\displaystyle #2$}}

\usepackage{}
\usepackage[T1]{fontenc}
\DeclareMathOperator{\colim}{colim}
\usepackage{quiver}

\begin{document}
\begin{abstract}\color{brown}
    \textit{Sequential Spectra}
\end{abstract}

We work in the category $\mathcal{T}$ of based( compactly generated weak Hausdorff ) spaces  and basepoint preserving maps. 
\begin{definition}
    A (sequential) spectrum $X$ is a sequence of spaces $X_n$ for $n\geq 0$ and structure maps $\sigma_n:\Sigma X_n=X_n\wedge S^1\to X_{n+1 }$. 
  
    A map of spectra $f:X\to Y$ is a sequence of maps $f_n:X_n\to Y_n$ such that each square of the following form commutes \marginnote{$S^1$ commutes\\ with $\wedge$}
    \[
        \begin{tikzcd}
\Sigma X_n \arrow[r, "\sigma_n"] \arrow[d, "\Sigma f_n"'] & X_{n+1} \arrow[d, "\Sigma f_{n+1}"] \\
\Sigma Y_n \arrow[r, "\sigma_n"']                         & Y_{n+1}                            
\end{tikzcd}
    \]

    The category of spectra and the maps described above is denoted by $\mathcal{S}p$.
\end{definition}
Note that the bonding maps $\sigma_i:\Sigma X_i\to \Sigma_{i+1 }$ have adjoints $\tilde{\sigma}_i:\Omega X_i\to \Omega X_{i+1}$ and we could equivalently define a spectrum using the adjoint bonding maps. 

\begin{definition}
    The degree $k\in \mathbb{Z}$ stable homotopy group of a spectrum $X$ is the abelian group 
    \[
        \pi_k(X)=\colim_n \pi_{k+n}(X_n)
    \]Here $\pi_{k+n }(X_n)\to \pi_{k+n+1 (X_{n+1})}$ maps the homotopy class of $\phi:S^{k+n } \to X_n$ to the class of $\sigma(S^1\wedge \phi):S^1\wedge S^{k+n } \to X_{n+1 }$.
\end{definition}

\begin{definition}
    A map of spectra $f:X\to Y$ is a stable equivalence ($\pi_*$ isomorphism ) if the induced map $f_*:\pi_*(X)\to \pi_*(Y)$ is an isomorphism. 
\end{definition}

Consider $\mathcal{W}\subset \mathcal{S}p$ to be the wide subcategory of stable equivalences. The stable homotopy category $\mathrm{ho } \mathcal{S}p$ is the localization of $\mathcal{S}p$ at $\mathcal{W}$. The functor 
\[
    \mathcal{S}p\to \mathcal{S}p[\mathcal{W}^{-1 } ]=\mathrm{ho } \mathcal{S}p
\] sends stable equivalences to isomorphisms. \marginnote{Initial functor?}

\begin{example}
    \begin{enumerate}
    \item The suspension spectrum $\Sigma^{\infty } X$ of a space $X$ is the spectrum, with $(\Sigma^{\infty } X)_n=\Sigma^nX$, with the bonding maps $\operatorname*{id }:S^1\wedge \Sigma^iX\to \Sigma^{i+1 } X$. The suspension spectrum for an unpointed space $X$ is obtained by adding a disjoint basepoint: $\Sigma^{\infty } _+X\coloneqq \Sigma^{\infty } (X_+)$. \\
    The homotopy groups of $\Sigma^{\infty } X$ are the stable homotopy groups of the space $X$. 
        \item An important example of a suspension spectrum is the \textit{sphere spectrum} $$\mathbb{S}\coloneqq \Sigma^{\infty } S^0$$ This is the sequence of spheres $S^0,S^1,S^2,\dots$ \\
        The homotopy groups of the sphere spectrum are called the \textit{stable stems }. They are the stable homotopy groups of $S^0$, often written $\pi_i^s\coloneqq \pi_i \mathbb{S}$.
    \item \label{2}The zero spectrum $\Sigma^{\infty}*$ is the suspension spectrum of a point, with the bonding maps 
    \[
        \Sigma(*)\cong * \to *
    \] 
    Every spectrum $X$ admits a unique maps of spectra $*\to X\to *$. $X$ is weakly contractible if one (equivalently both) of these maps are stable equivalences. \\
    The suspension spectrum has all of its homotopy groups zero. 

    Given two spectra $X$ and $Y$, the zero map $X\to Y$ is the unique map of spectra that factors through $*$
    \[
        \begin{tikzcd}
X_n \arrow[rd] \arrow[rr] &              & Y_n \\
                          & * \arrow[ru] &    
\end{tikzcd}
    \]
    
    \item\label{3} Eilenberg-Maclane spaces are infinite loopspaces. We can define teh Eilenberg-Maclane spectrumm $HA$ by 
    \[
        (HA)_n=K(A,n)
    \] with bonding maps adjoing to the canonical equivalences $K(A,n)\simeq \Omega K(A,n+1)$. We use the letter $H$ because the infinite loopspace represents cohomology with coefficients in $A$. \marginnote{todo}

    The homotopy grops of $HA$ are given by 
    \[
        \pi_i HA=\begin{cases}
            A&i=0\\
            0&i\neq 0
        \end{cases}
    \]  
    \item \label{4} The complex $K$ theory spectrum $KU$
    \[
        KU_n=\begin{cases}
            \mathbb{Z}\times BU&n \text{ even}\\
            U &n \text{ odd}
        \end{cases}
    \]The structure maps adjoint to the equivalences 
    \[
        \Omega U\simeq \mathbb{Z}\times BU \quad \Omega(\mathbb{Z}\times BU)\simeq U
    \] 
    The homotopy groups of $KU$ are given by \marginnote{Bott element\\generator    }
    \[
        \pi_iKU=\begin{cases}
            \mathbb{Z}&i \text{ even}\\
            0& i \text{ odd}
        \end{cases}
    \]\end{enumerate}
\end{example}
\begin{definition}
    An $\Omega-$ spectrum (or fibrant spectrum) is a spectrum $X$ in which the adjunct bonding maps 
    \[
        X_n\xrightarrow[ ]{\simeq } \Omega X_{n+1 }
    \] are weak homotopy equivalences. 
\end{definition} The zeroeth space $X_0$ of an $\Omega$ -spectrum is an infinite loop space. 
Examples \ref{2},\ref{3},\ref{4} above are $\Omega$-spectra. 

\begin{proposition}
    If $X$ is a spectrum, there exists an $\Omega -$spectrum $RX$ with a stable equivalence 
    \[
        X\to RX
    \] which is natural. 
\end{proposition}

$R$ here is a right deformation of the category $\mathcal{S}p$. \marginnote{homotopy \\category?}

For a spectrum $X$, with spaces $X_n$ and bonding maps $\sigma_n$, $RX$ is defined as 
\[
    (RX)_n=\operatorname*{hocolim }(X_n\xrightarrow[ ]{\sigma_n}\Omega X_{n+1}\xrightarrow[ ]{\Omega \sigma_{n+1 }}\Omega^{2 } X_{n+2 }\xrightarrow[ ]{ }\cdots ) 
\]The commuting maps 
\[\begin{tikzcd}
	X & {\Omega X_{n+1}} & {\Omega^2 X_{n+2}} & {} \\
	{\Omega X_{n+1}} & {\Omega^2 X_{n+2}} & {\Omega^3 X_{n+3}} & \cdots
	\arrow["{\sigma_n}"', from=1-1, to=2-1]
	\arrow["{\sigma_n}", from=1-1, to=1-2]
	\arrow["{\mathrm{id}}", dashed, from=1-2, to=2-1]
	\arrow["{\Omega \sigma_{n+1}}"', from=2-1, to=2-2]
	\arrow["{\Omega \sigma_{n+1}}", from=1-2, to=2-2]
	\arrow["{\Omega \sigma_{n+1}}", from=1-2, to=1-3]
	\arrow[from=2-2, to=2-3]
	\arrow[from=1-3, to=2-3]
	\arrow["{\mathrm{id}}", dashed, from=1-3, to=2-2]
	\arrow[from=1-3, to=1-4]
	\arrow[from=2-3, to=2-4]
\end{tikzcd}\]
induce a map of homotopy colimits
\[
    \operatorname*{hocolim }_{m\to \infty}\Omega^mX_{n+m }\xrightarrow[ ]{\simeq }\operatorname*{hocolim }_{m\to \infty }\Omega^{1+m } X_{n+1+m }\xrightarrow[ ]{\simeq }\Omega (\operatorname*{hocolim }_{m\to \infty      }\Omega^mX_{n+1+m})
\]We define the bonding maps $(RX)_n\xrightarrow[ ]{\simeq }\Omega(RX)_{n+1}$ to be the composite map from above. 

\begin{definition}
    For any spectrum $X$, we define its infinite loop space to be 
    \begin{align*}
        \Omega^{\infty}: \mathcal{S}p&\to \mathcal{T}\\
        X&\mapsto (RX)_0
    \end{align*}

     For example, $\Omega^{\infty }\mathbb{S}=\colim \Omega^n \mathbb{S} ^n$. 
\end{definition}
\begin{proposition}
    The functors $\Omega^{\infty}$ and $\Sigma^{\infty }$ are adjoint 
  \[\begin{tikzcd}
	{\mathcal{S}p} & {} & {\mathcal{T}}
	\arrow[""{name=0, anchor=center, inner sep=0}, "{\Omega^{\infty}}", curve={height=-12pt}, from=1-1, to=1-3]
	\arrow[""{name=1, anchor=center, inner sep=0}, "{\Sigma^{\infty}}", curve={height=-12pt}, from=1-3, to=1-1]
	\arrow["\dashv", draw=none, from=0, to=1]
\end{tikzcd}\]
\end{proposition}\marginnote{remark about QX}

\begin{definition}
    If $X\in \mathcal{S}p$ and $K\in \mathcal{T}$, then we can define the tensor or smash product $K\wedge X$ by lettign $(K\wedge X)_n=K\wedge X_n$ and the bonding maps $K\wedge X_n\wedge S^1\to K\wedge X_{n+1 }$. 

We define \begin{align*}
    \Sigma:\mathcal{S}p&\to \mathcal{S}p\\
    X&\mapsto S^1\wedge X
\end{align*}

\end{definition}

\begin{definition}

    If $X$ is a spectrum and $K$ is a based space, we form the cotensor or function spectrum $F(K,X)$ by applying $\mathrm{Map } _*(K,-)$ to every spectrum level of $X$. So $F(K,X)$ is the spectrum whose $n$th level is the space of based maps $\mathrm{ Map } _*(K,X_n)$. The bonding maps are 
    \[
         \mathrm{ Map } _*(K,X_n)\xrightarrow[ ]{\mathrm{ Map } _*(K, \Omega X_{n+1 })}\mathrm{Map } _*(K, \Omega X_{n+1})\xleftrightarrow[]{ \cong }\Omega \mathrm{Map } _*(K,X_{n+1 })
    \]
    
    As a special case, we define 
    \begin{align*}
        \Omega: \mathcal{S}p&\to \mathcal{S}p\\
        X&\mapsto F(S^1,X)
    \end{align*}
\end{definition}

These two operations are adjoint functors on spectra. 

\begin{proposition}\marginnote{PROOF }
    For any spectrum $X$ , there are isomorphisms 
    \[
        \pi_{k+1 }(\Sigma X)\cong \pi_k(X)\cong \pi_{k-1 }(\Omega X)
    \]
\end{proposition}
\begin{corollary}
    There are natural stable equivalences $X\to \Sigma \Omega X$ and $\Sigma \Omega X\to X$. 
\end{corollary}Thus $\Sigma$ and $\Omega$ are inverse functors upto stable equivalence. 

\begin{definition}
    A cofibers sequence is a sequence of the form 
    \[
        X \xrightarrow[ ]{f }Y\to C_f
    \]where $Cf$ is the homotopy cofiber, which is defined as 
    \[
        (Cf)_n=\text{homotopy cofiber of }(X \xrightarrow[ ]{f_n }Y_n)
    \]

    Similarly a fiber sequence is anything of the form 
    \[
         Fg\to Y \xrightarrow[ ]{g }Z
    \]where $Fg$ is the homotopy fiber. 
\end{definition}

\begin{theorem}
     For each fiber sequence $X\to Y\to Z$, we get a long exact sequence
     \[
        \dots\to [W,X] \xrightarrow[ ]{f_* }[W,Y]\xrightarrow[ ]{g_* }[W,Z]\to \dots
     \]
\end{theorem}

\begin{theorem}\marginnote{check proof}
     For each cofiber sequence $X\to Y\to Z$, we get a long exact sequence
     \[
       \cdots \leftarrow [X,W]\leftarrow[Y,W]\leftarrow[Z,W]\leftarrow \cdots 
     \]
     and also a long exact sequence
     \[
        [W,X]\xrightarrow[ ]{f_* }[W,Y]\xrightarrow[ ]{i_* }[W,Z]
     \]
\end{theorem}
\begin{proof}
     Suppose we have a map $g: W \rightarrow Y$ so that $i g \simeq 0$. Let $h$ be a null homotopy of this composite. This is precisely a map $b: C W \rightarrow C f$. We thus have the following diagram, where the horizontal parts are cofibre sequences,
\[\begin{tikzcd}
	X & Y & Cf & {\Sigma X} & {\Sigma Y} & \cdots \\
	W & W & CW & {\Sigma W} & {\Sigma W} & \cdots
	\arrow["{-\Sigma^{-1} j}"', tail reversed, no head, from=1-1, to=2-1]
	\arrow["g"', tail reversed, no head, from=1-2, to=2-2]
	\arrow["h"', tail reversed, no head, from=1-3, to=2-3]
	\arrow["k"', dashed, tail reversed, no head, from=1-4, to=2-4]
	\arrow["{\Sigma g}"', tail reversed, no head, from=1-5, to=2-5]
	\arrow["f", from=1-1, to=1-2]
	\arrow["i", from=1-2, to=1-3]
	\arrow["{-1}"', from=2-4, to=2-5]
	\arrow["1"', from=2-1, to=2-2]
	\arrow["i"', from=2-2, to=2-3]
	\arrow[from=2-3, to=2-4]
	\arrow[from=1-3, to=1-4]
	\arrow[from=1-4, to=1-5]
	\arrow[from=1-5, to=1-6]
	\arrow[from=2-5, to=2-6]
\end{tikzcd}\]
Because these are cofibre sequences, we get the dashed arrow $k$ making the diagram commute for free, and the rest is automatic. But $\Sigma$ has an inverse in SHC. Applying $\Sigma^{-1}$ gives us a map $\Sigma^{-1} k: W \rightarrow X$, making the necessary diagram commute (functoriality of $\Sigma^{-1}$ ). This proves exactness.

\end{proof}

\begin{lemma}
    For each map of spectra $f:X\to Y$ there is a natrual stable equivalence 
    \[
        \varepsilon:\Sigma Ff\to Cf
    \] or equivalently its adjoint. 
\end{lemma}
\begin{proof}
     We define $\varepsilon$ by
     \begin{align*}
        \varepsilon: \Sigma(X\times_Y PY )&\to Y\cup_X CX \\
            (t,x,\gamma)&=\begin{cases}
                \gamma(2t)&t\leq 1/2 \\
                (x,2-2t)& t\geq 1/2
            \end{cases}
     \end{align*}
We check the commutativity upto homotopy of the following squares by looking  at the suspension of the fiber sequence of $f$ and cofiber sequence of $f$. 
     \[\begin{tikzcd}
	{\Sigma\Omega X} & {\Sigma\Omega Y} & {\Sigma(X\times_Y PY)} & {\Sigma X} & {\Sigma Y} \\
	X & Y & {Y\cup_X CX} & {\Sigma X} & {\Sigma Y}
	\arrow["{\Sigma\Omega f}", from=1-1, to=1-2]
	\arrow[from=1-2, to=1-3]
	\arrow["f"', from=2-1, to=2-2]
	\arrow[from=1-1, to=2-1]
	\arrow[from=1-2, to=2-2]
	\arrow["\varepsilon", from=1-3, to=2-3]
	\arrow[from=2-2, to=2-3]
	\arrow[from=1-3, to=1-4]
	\arrow[from=2-3, to=2-4]
	\arrow["{\text{flip}}"', from=1-4, to=2-4]
	\arrow["{\text{flip}}", from=1-5, to=2-5]
	\arrow["{\Sigma f}"', from=2-4, to=2-5]
	\arrow["{\Sigma f}", from=1-4, to=1-5]
\end{tikzcd}\]This gives that $\varepsilon$ is a $\pi_*$ isomorphism. 
\end{proof}

\begin{proposition}
    $X\to Y\to Z$ is a cofiber sequence iff it is a fiber sequence. 
\end{proposition}

\begin{proof}
     We look at the squares in the diagram 
     \[\begin{tikzcd}
	X & Y & Cf & {\Sigma X} & {\Sigma Y} \\
	X & Y & Z & Cg & {\Sigma Y} \\
	\\
	&&& {\Sigma (Y\times _Z F(I,Z))}
	\arrow[Rightarrow, no head, from=1-2, to=2-2]
	\arrow["{h\cup g}", from=1-3, to=2-3]
	\arrow[Rightarrow, no head, from=1-5, to=2-5]
	\arrow["f", from=1-1, to=1-2]
	\arrow["g"', from=2-2, to=2-3]
	\arrow[from=2-1, to=2-2]
	\arrow[from=1-2, to=1-3]
	\arrow[from=1-3, to=1-4]
	\arrow[from=1-4, to=1-5]
	\arrow["{\Sigma(f\times h)}"', curve={height=18pt}, from=1-4, to=4-4]
	\arrow["{h\cup Cf}", from=1-4, to=2-4]
	\arrow[from=2-3, to=2-4]
	\arrow[from=2-4, to=2-5]
	\arrow["\varepsilon"',"\simeq", from=4-4, to=2-4]
\end{tikzcd}\] We have
\begin{align*}
    X\to Y\to Z \text{ a cofiber sequence}&\iff h\cup g \text{ is a stable equivalence}\\
        &\iff h\cup Cf \text{ is a stable equivalence   }\\
        &\iff \Sigma(f\times h) \text{ is a stable equivalence  }\\
        &\iff f\times h \text{ is a stable equivalence}\\
        &\iff X\to Y\to Z \text{ is a fiber sequence} \\
\end{align*}

\end{proof}
\begin{lemma}
    A commuting square of spectra is a homotopy pushout iff it is a homotopy pullback. 
\end{lemma}

\begin{proposition}\marginnote{just this?}
    Finite coproducts and finite products coincide in $\mathcal{S}p$
    \[
        X\vee Y\cong X\times Y 
    \]
\end{proposition}
\end{document}